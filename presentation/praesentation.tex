\documentclass[12pt, draft]{beamer}
\usetheme{Darmstadt}

\usepackage[utf8]{inputenc}
\usepackage[german]{babel}
\usepackage[T1]{fontenc}
\usepackage{amsmath}
\usepackage{amsfonts}
\usepackage{amssymb}
\usepackage{tikz}
\usepackage{tabularx}
\usepackage{pgfpages}
\usepackage{listings}
\usepackage{color}
\usepackage[edges]{forest}

\setbeameroption{show notes on second screen}

\title{Implementation eines Blokus-Spielers}
\author{\mbox{Lukas Jagemann}}
\date{\today}

\lstset{
	aboveskip=1mm,
	belowskip=1mm,
	showstringspaces=false,
	columns=flexible,
	basicstyle={\small\ttfamily},
	numbers=none,
	breaklines=true,
	breakatwhitespace=true,
	tabsize=4
}
% https://tex.stackexchange.com/questions/89574/language-option-supported-in-listings
\lstdefinelanguage{JavaScript}{
	keywords={typeof, new, true, false, catch, function, return, null, catch, switch, var, if, in, while, do, else, case, break},
	keywordstyle=\color{blue}\bfseries,
	ndkeywords={class, export, boolean, throw, implements, import, this},
	ndkeywordstyle=\color{darkgray}\bfseries,
	identifierstyle=\color{black},
	sensitive=false,
	comment=[l]{//},
	morecomment=[s]{/*}{*/},
	commentstyle=\color{purple}\ttfamily,
	stringstyle=\color{red}\ttfamily,
	morestring=[b]',
	morestring=[b]"
}

\begin{document}

\begin{frame}
    \begin{figure}[h]
        \centering
        \resizebox{0.5\linewidth}{!}{\includegraphics{media/blokus.png}}
    \end{figure}
    %\vspace*{-20pt}
    \titlepage
\end{frame}

\begin{frame}{Inhalt}
	\note{...}
	\setcounter{tocdepth}{1}
    \tableofcontents
\end{frame}

\section{Blokus}
\subsection{Überblick}
\begin{frame}
	Entstanden $\sim$2000\\
	Ursprünglich für 4 Spieler
    \begin{figure}[tp]
        \centering
        \resizebox{0.7\linewidth}{!}{\includegraphics[width=\linewidth]{media/blokus4p.jpg}}\\
        \tiny Quelle: \url{https://media.takealot.com/covers\_tsins/32082437/81LEjjVViUL.\_SL1500\_-zoom.jpg}
    \end{figure}
\end{frame}
\begin{frame}
	\begin{tabularx}{\hsize}{*2{>{\centering\arraybackslash}X}}
		\includegraphics[width=\linewidth]{media/blokus3p.jpg}\linebreak
		\tiny Quelle: \url{http://www.passionforpuzzles.com/weblog/wp-content/uploads/2010/12/613KSlRx9oS._SL1500_.jpg}
	    &
		\includegraphics[width=\linewidth]{media/blokus2p.jpg}\linebreak
		\tiny Quelle: \url{http://boardgaming.com/wp-content/uploads/2012/02/blokus-duo-game-in-play.jpg}
	\end{tabularx}
\end{frame}

\subsection{Wie man startet}
\begin{frame}
	\includegraphics[width=0.8\linewidth]{media/how2play0.png}
\end{frame}
\begin{frame}
	\includegraphics[width=0.8\linewidth]{media/how2play1.png}
\end{frame}

\subsection{Wie man spielt}
\begin{frame}
	\begin{tabular}{c c}
		\huge \color{green} $\checkmark$ & \\
		\includegraphics[width=0.5\linewidth]{media/how2play2.png}
		&
	\end{tabular}
\end{frame}
\begin{frame}
	\begin{tabular}{c c}
		\huge \color{green} $\checkmark$ & \huge \color{green} $\checkmark$\\
		\includegraphics[width=0.5\linewidth]{media/how2play2.png}
		&
		\includegraphics[width=0.5\linewidth]{media/how2play3.png}
	\end{tabular}
\end{frame}
\begin{frame}
\begin{tabular}{c c}
	\huge \color{red} $\times$ & \\
	\includegraphics[width=0.5\linewidth]{media/how2play4.png}
	&
\end{tabular}
\end{frame}
\begin{frame}
\begin{tabular}{c c}
	\huge \color{red} $\times$ & \huge \color{green} $\checkmark$\\
	\includegraphics[width=0.5\linewidth]{media/how2play4.png}
	&
	\includegraphics[width=0.5\linewidth]{media/how2play5.png}
\end{tabular}
\end{frame}

\subsection{Wie man gewinnt}
\begin{frame}
	\note{Begriff 'Bits' erklären!}
	Spieler mit den meisten 'Bits' auf dem Feld gewinnt.
	\includegraphics[width=0.7\linewidth]{media/how2play6.png}
\end{frame}

\subsection{Online Ressourcen}
\begin{frame}
	\note{Strategy guter Einstieg\\ Pentolla online spielen. Aktive community!\\Eigenwerbung}
	\begin{itemize}
		\item \url{http://blokusstrategy.com/}
		\item \url{http://pentolla.com/play/}
		\item und jetzt auch: \url{http://splamy.com/blokus/}
	\end{itemize}
\end{frame}


\section{Das Projekt}
\subsection{Wunderschönes CSS3(LESS) und HTML5}
\begin{frame}
	\frametitle{HTML5}
	\begin{itemize}
		\note{Alternative: SVG; Aufwendige Effekte\\Alternative: Canvas2D; Canvas selber zeichnen, Reihenfolge beachten, clickevents selber umrechen\\(ein kandidat mit svg), (einer mit canvas) (zwei plain html)}
		\item Valides HTML5 !
		\item Beste Kompatibilität zwischen aktuellen Browsern
		\item Kurzzeitig Canvas2D, dann doch 'plain' HTML
	\end{itemize}
\end{frame}
\begin{frame}
	\note{Farben in echtzeit änderbar\\ecken mit transformationen}
	\frametitle{CSS3}
	\begin{itemize}
		\item Einfache hardware-beschleunigte Effekte:\\
			\includegraphics{media/beautiful0.png}
		\pause
		\item Transformationen und Animationen:
			\includegraphics{media/beautiful1.png}
	\end{itemize}
\end{frame}
\begin{frame}
	\frametitle{LESS}
	\note{Alternative SCSS, SASS,...\\Hust, alle Mühen vergebens; Seite sieht trozdem miserabel aus.}
	\begin{itemize}
		\item LESS ist ein CSS precompiler
		\item Praktische Funktionen wie z.B. Variablen und Farbmanipulationen
	\end{itemize}
% TODO vielleicht noch ein Bild hier, aber nicht nötig
\end{frame}

\subsection{TypeScript, das bessere JavaScript}
\begin{frame}[fragile]
	\frametitle{TypeScript}
	\note{JS ist valides TS\\TS starke Empfehlung nicht plain JS zu schreiben\\ES6 implementiert viele TS features\\Verhindert viele allgemeine Fehler\\IDE error support\\IDE syntax vervollständigung\\Einheitliche kompilierung nach ES3, ES5, ES6,...}
	TypeScript...
	\begin{itemize}
		\item ist eine Untermenge von JavaScript
		\item wird transpiliert zu JavaScript
		\item hat Klassen:\\
		\begin{lstlisting}[language=JavaScript]
class A {
	public Prop: boolean;
	private field: number;
}
		\end{lstlisting}
		\item ...mit Vererbung
		\item hat ein strenges Typensystem:\\
		\begin{lstlisting}[language=JavaScript]
let a: boolean = 5; // FEHLER!
		\end{lstlisting}
		\item ...und vieles mehr
	\end{itemize}
\end{frame}

\subsection{Code Struktur}
\begin{frame}
	\begin{itemize}
		\item Modular aufgeteilt in Klassen
		\item Mit Bedacht auf Erweiterbarkeit entworfen:\\ \vspace*{5pt}
			\includegraphics[width=0.5\linewidth]{media/structure1.eps}
	\end{itemize}
\end{frame}
\begin{frame}
	Z.b.\\ \vspace*{5pt}
	\includegraphics[width=0.8\linewidth]{media/structure2.eps}
\end{frame}

\subsection{GameState}
\begin{frame}
	Ein GameState beschreibt einen kompletten Spielzustand:
	\begin{itemize}
		\item Wer ist dran
		\item Welche Steine sind noch verfügbar
		\item Wie sieht das Spielfeld aus
		\item (Caching)
	\end{itemize}
\end{frame}
\begin{frame}
	\note{=>History slider\\Sehr wichtig beim lernen\\noch wichtiger beim debuggen :P\\Serializierbarkeit}
	Aber warum der Extraaufwand?\\ \vspace*{5pt}
	\includegraphics[width=\linewidth]{media/gamestate1.png}
\end{frame}

\section{Minimax}
\subsection{Wie funktioniert MiniMax}
\begin{frame}
	\note{Minimax macht keine riskanten züge\\Für informatiker wachsen Bäume falsch rum :P}
	Eigenschaften von MiniMax:
	\begin{itemize}
		\item basiert darauf das \emph{beide} Spieler optimal spielen
		\item lässt sich rekursiv beliebig stark machen (theoretisch)
		\item spielt 'sicher'
	\end{itemize}
	Wie MiniMax funktioniert:\\
	Für jede Rekursionsstufe wird abwechselnd für einen eigenen Zug die beste und einen gegnerischen Zug die für einen schlechteste Möglichkeit gewählt. Dafür werden die Knoten auf der untersten Ebene ausgewertet und der Baum von unten nach oben reduziert.\\
\end{frame}

% Ziel: beide spieler spielen optimal
\subsection{Grafische Veranschaulichung}
\begin{frame}
	Wertung: $Bits_{own}-Bits_{enemy}$\\
	\pause
	\includegraphics[width=\linewidth]{media/mm1.png}
\end{frame}

\begin{frame}
	\includegraphics[width=\linewidth]{media/mm3.png}
\end{frame}

\begin{frame}
	\includegraphics[width=\linewidth]{media/mm4.png}
\end{frame}

\begin{frame}[fragile]
\forestset{
	declare toks={graphics}{},
}
\begin{forest}
	before typesetting nodes={
		for tree={
			content/.wrap 2 pgfmath args={\includegraphics[#2]{#1}}{content()}{graphics()},
		},
		where={isodd(n_children())}{calign=child, calign child/.wrap pgfmath arg={#1}{int((n_children()+1)/2)}}{},
	},
	for tree={
		parent anchor=children,
		child anchor=parent,
		edge={line width=2.0pt},
	},
	[media/mm1.png, graphics={scale=0.4}
		[media/mm2.png, graphics={scale=0.4}]
		[media/mm3.png, graphics={scale=0.4}]
	]
\end{forest}
\end{frame}

\begin{frame}[fragile]
\forestset{
	declare toks={graphics}{},
}
\begin{forest}
	before typesetting nodes={
		for tree={
			content/.wrap 2 pgfmath args={\includegraphics[#2]{#1}}{content()}{graphics()},
		},
		where={isodd(n_children())}{calign=child, calign child/.wrap pgfmath arg={#1}{int((n_children()+1)/2)}}{},
	},
	for tree={
		parent anchor=children,
		child anchor=parent,
		edge={line width=2.0pt},
	},
	[media/mm1.png, graphics={scale=0.1}
		[media/mm2.png, graphics={scale=0.2}
			[media/mm7.png, graphics={scale=0.3}
				[media/mm8.png, graphics={scale=0.3}]
			]
		]
		[media/mm3.png, graphics={scale=0.2}]
	]
\end{forest}
\end{frame}

\begin{frame}[fragile]
\forestset{
	declare toks={graphics}{},
}
\hspace*{-1cm}
\begin{forest}
	before typesetting nodes={
		for tree={
			content/.wrap 2 pgfmath args={\includegraphics[#2]{#1}}{content()}{graphics()},
		},
		where={isodd(n_children())}{calign=child, calign child/.wrap pgfmath arg={#1}{int((n_children()+1)/2)}}{},
	},
	for tree={
		parent anchor=children,
		child anchor=parent,
		edge={line width=2.0pt},
	},
	[media/mm1.png, graphics={scale=0.1}
		[media/mm2.png, graphics={scale=0.25}
			[media/mm7.png, graphics={scale=0.3}
				[media/mm8.png, graphics={scale=0.3}]
			]
		]
		[media/mm3.png, graphics={scale=0.25}
			[media/mm4.png, graphics={scale=0.3}]
			[media/mm5.png, graphics={scale=0.3}
				[media/mm6.png, graphics={scale=0.3}]
			]
		]
	]
\end{forest}
\end{frame}

\begin{frame}[fragile]
\forestset{
	declare toks={graphics}{},
}
\hspace*{-1cm}
\begin{forest}
	before typesetting nodes={
		for tree={
			content/.wrap 2 pgfmath args={\includegraphics[#2]{#1}}{content()}{graphics()},
		},
		where={isodd(n_children())}{calign=child, calign child/.wrap pgfmath arg={#1}{int((n_children()+1)/2)}}{},
	},
	for tree={
		parent anchor=children,
		child anchor=parent,
		edge={line width=2.0pt},
	},
	[media/mm1.png, graphics={scale=0.1}
		[media/mm2.png, graphics={scale=0.25}
			[media/mm7.png, graphics={scale=0.3}
				[media/mm8.png, graphics={scale=0.3}, edge label={node[midway,auto,font=\small]{3}} ]
			]
		]
		[media/mm3.png, graphics={scale=0.25}
			[media/mm4.png, graphics={scale=0.3}, edge label={node[midway,auto,font=\small]{-2}}]
			[media/mm5.png, graphics={scale=0.3}
				[media/mm6.png, graphics={scale=0.3}, edge label={node[midway,auto,font=\small]{2}} ]
			]
		]
	]
\end{forest}
\end{frame}

\begin{frame}[fragile]
\forestset{
	declare toks={graphics}{},
}
\hspace*{-1cm}
\begin{forest}
	before typesetting nodes={
		for tree={
			content/.wrap 2 pgfmath args={\includegraphics[#2]{#1}}{content()}{graphics()},
		},
		where={isodd(n_children())}{calign=child, calign child/.wrap pgfmath arg={#1}{int((n_children()+1)/2)}}{},
	},
	for tree={
		parent anchor=children,
		child anchor=parent,
		edge={line width=2.0pt},
	},
	[media/mm1.png, graphics={scale=0.1}
		[media/mm2.png, graphics={scale=0.25}
			[media/mm7.png, graphics={scale=0.3}, edge label={node[midway,auto,font=\small]{3}}
				[media/mm8.png, graphics={scale=0.3}, edge label={node[midway,auto,font=\small]{3}} ]
			]
		]
		[media/mm3.png, graphics={scale=0.25}
			[media/mm4.png, graphics={scale=0.3}, edge label={node[midway,auto,font=\small]{-2}}]
			[media/mm5.png, graphics={scale=0.3}, edge label={node[midway,auto,font=\small]{2}}
				[media/mm6.png, graphics={scale=0.3}, edge label={node[midway,auto,font=\small]{2}} ]
			]
		]
	]
\end{forest}
\end{frame}

\begin{frame}[fragile]
\forestset{
	declare toks={graphics}{},
}
\hspace*{-1cm}
\begin{forest}
	before typesetting nodes={
		for tree={
			content/.wrap 2 pgfmath args={\includegraphics[#2]{#1}}{content()}{graphics()},
		},
		where={isodd(n_children())}{calign=child, calign child/.wrap pgfmath arg={#1}{int((n_children()+1)/2)}}{},
	},
	for tree={
		parent anchor=children,
		child anchor=parent,
		edge={line width=2.0pt},
	},
	[media/mm1.png, graphics={scale=0.1}
		[media/mm2.png, graphics={scale=0.25}, edge label={node[midway,auto,font=\small]{3}}
			[media/mm7.png, graphics={scale=0.3}, edge label={node[midway,auto,font=\small]{3}}
				[media/mm8.png, graphics={scale=0.3}, edge label={node[midway,auto,font=\small]{3}} ]
			]
		]
		[media/mm3.png, graphics={scale=0.25}, edge label={node[midway,auto,font=\small]{-2}}
			[media/mm4.png, graphics={scale=0.3}, edge label={node[midway,auto,font=\small]{-2}}]
			[media/mm5.png, graphics={scale=0.3}, edge label={node[midway,auto,font=\small]{2}}
				[media/mm6.png, graphics={scale=0.3}, edge label={node[midway,auto,font=\small]{2}} ]
			]
		]
	]
\end{forest}
\end{frame}


\subsection{Wann MiniMax gut funktioniert}
\begin{frame}
    \begin{itemize}
      \item<1-| alert@4> Niedriger Verzweigungsfaktor
      \item<2-> Aussagekräftige Spielzustandsbewertungen
      \item<3-> MiniMax wird gegen Ende oft sehr stark
    \end{itemize}
\end{frame}

\subsection{...und wann eher weniger}
\begin{frame}
	\includegraphics[width=\linewidth]{media/mmexp1.png}
\end{frame}

\subsection{Abwandlungen}
\begin{frame}
	\note{TODO}
    \begin{itemize}
		\item<1-> Alpha–beta pruning
		\item<2-> Negamax
		\item<2-> MaxiMin
    \end{itemize}
\end{frame}
% aplha beta heuristiken

\section{Die Logik}
\subsection{Konzept}
\begin{frame}
	\note{TODO}
	\begin{itemize}
		\item<1-> Viel mehr Wert auf Heuristik legen
		\item<2-> Verschiedene Bewertungen kombinieren
		\item<3-> %TODO
    \end{itemize}
\end{frame}

\subsection{Gewichtung: Bits}
\begin{frame}
	Mehr Steine = Besser!\\
	\pause
	Gewichtung: $Bits_{own}$\\
	\pause
	\includegraphics[width=\linewidth]{media/wgh1.png}
\end{frame}

\subsection{Gewichtung: Ecken}
\begin{frame}
	\note{Dead Corner erklären}
	Mehr Ecken = Mehr Möglichkeiten!\\
	\pause
	Gewichtung: $Corner_{own}$\\
	\pause
	\includegraphics[width=0.6\linewidth]{media/wgh2.png}
\end{frame}
\begin{frame}
	Mehr Ecken = Mehr Möglichkeiten!\\
	Gewichtung: $Corner_{own}$\\
	\includegraphics[width=0.6\linewidth]{media/wgh3.png}
\end{frame}
\begin{frame}
	Mehr eigene Ecken, weniger gegnerische Ecken!\\
	\pause
	Gewichtung: $Corner_{own} - Corner_{enemy}$\\
	\pause
	\includegraphics[width=0.6\linewidth]{media/wgh4.png}
\end{frame}
\begin{frame}
	Mehr eigene Ecken, weniger gegnerische Ecken!\\
	Gewichtung: $Corner_{own} - Corner_{enemy}$\\
	\includegraphics[width=0.6\linewidth]{media/wgh5.png}
\end{frame}

\subsection{Gewichtung: TrueLength}
\begin{frame}
	\note{True Length, genial! Moment mal was ist das eig?\\Erklären...\\Und wofür brauchen wir das?\\-> Effektivität eines Steins (Nächste Folie)}
	True Length/Rubik Length!?\\
	\pause
	Gewichtung: $TrueLength_{own}$\\
	\pause
	\includegraphics[width=0.8\linewidth]{media/wgh6.png}\\
	\tiny Quelle: \url{http://blokusstrategy.com/early-pieces-and-rubik-distance/}
\end{frame}
\begin{frame}
	True Length/Rubik Length!\\
	Gewichtung: $TrueLength_{own}$\\
	\includegraphics[width=0.8\linewidth]{media/wgh7.png}\\
	\tiny Quelle: \url{http://blokusstrategy.com/early-pieces-and-rubik-distance/}
\end{frame}
\begin{frame}
	True Length/Rubik Length!\\
	Gewichtung: $Corner_{own} + TrueLength_{own}$\\
	\pause
	\includegraphics[width=0.6\linewidth]{media/wgh8.png}
\end{frame}
\begin{frame}
	True Length/Rubik Length!\\
	Gewichtung: $Corner_{own} + TrueLength_{own}$\\
	\includegraphics[width=0.6\linewidth]{media/wgh9.png}
\end{frame}

\subsection{Gewichtung: Weitestes Feld}
\begin{frame}
	Weitestes Feld\\
	\pause
	Gewichtung: $Bits_{own} - FurthestField_{own}$\\
	\pause
	\includegraphics[width=0.6\linewidth]{media/wgh10.png}
\end{frame}
\begin{frame}
	\note{Hier is was komisches passiert! Folie gleich}
	Weitestes Feld\\
	Gewichtung: $Bits_{own} - FurthestField_{own}$\\
	\only<1>{\includegraphics[width=0.6\linewidth]{media/wgh11.png}}
	\only<2>{\includegraphics[width=0.6\linewidth]{media/wgh12.png}}
\end{frame}
\begin{frame}
	Mein weitestes Feld <> generisches weitestes Feld\\
	\pause
	Gewichtung: $Bits_{own} - FurthestField_{own} + FurthestField_{enemy}$\\
	\pause
	\includegraphics[width=0.6\linewidth]{media/wgh13.png}
\end{frame}
\begin{frame}
	Mein weitestes Feld <> generisches weitestes Feld\\
	Gewichtung: $Bits_{own} - FurthestField_{own} + FurthestField_{enemy}$\\
	\includegraphics[width=0.6\linewidth]{media/wgh14.png}
\end{frame}

\subsection{Gewichtung: Erreichbare Felder}
\begin{frame}
	\note{Wiederverwertung von vorheriger Berechnung}
	Erreichbare Felder = gute Felder!\\
	\pause
	Gewichtung: $AccesibleSpace_{own}$ bzw. $- DeadSpace_{own}$
	\pause
	\includegraphics[width=0.6\linewidth]{media/wgh15.png}
\end{frame}
\begin{frame}
	\note{Wo man nicht hinkommt kann man nicht bestaft werden einen so langen Weg hin zu brauchen.}
	Erreichbare Felder = gute Felder!\\
	Gewichtung: $AccesibleSpace_{own}$
	\only<1>{\includegraphics[width=0.6\linewidth]{media/wgh16.png}}
	\only<2>{\includegraphics[width=0.6\linewidth]{media/wgh17.png} \vspace{2cm} \Huge =26}
	\only<3>{\includegraphics[width=0.6\linewidth]{media/wgh18.png} \vspace{2cm} \Huge =7}
\end{frame}
\begin{frame}
	Erreichbare Felder = gute Felder!\\
	Gewichtung: $Bits_{own} - FurthestField_{own} + AccesibleSpace_{own}$
	\pause
	\includegraphics[width=0.6\linewidth]{media/wgh19.png}
\end{frame}
\begin{frame}
\note{Schon gar nicht schlecht\\i1 völlig falsch\\i1 vor dem i2\\Kann man ihm aber nich übel nehmen bei der Rechentiefe}
	Erreichbare Felder = gute Felder!\\
	Gewichtung: $Bits_{own} - FurthestField_{own} + AccesibleSpace_{own}$
	\includegraphics[width=0.6\linewidth]{media/wgh20.png}
\end{frame}
\begin{frame}
	\note{Genial, nicht nur seinen Weg gesischert, sonder auch uns den ganzen Weg abgeschnitten}
	Erreichbare Felder = gute Felder!\\
	Gewichtung: $Bits_{own} - FurthestField_{own} + AccesibleSpace_{own} - AccesibleSpace_{enemy}$\\
	\pause
	\includegraphics[width=0.6\linewidth]{media/wgh21.png}
\end{frame}

\subsection{Gewichtung: Sieg}
\begin{frame}
Normalerweise wird bei MiniMax ein Sieg mit $+\infty$ und eine Niederlage mit $-\infty$ bewertet.
\pause
\includegraphics[width=0.6\linewidth]{media/wgh22.png}
\end{frame}
\begin{frame}
Stattdessen wird hier mit $+10000$ und $-10000$ bewertet.
\pause
\includegraphics[width=0.6\linewidth]{media/wgh23.png}
\end{frame}

\subsection{Gewichtung finden}
\begin{frame}
	\note{Alles in einen Topf werfen, umrühren, und sich fragen warum die AI genauso schlecht spielt wie vorher!}
	\only<1-3>{
		\pause
		Aktuelle Gewichtung:\\$Bits_{own} \linebreak + Corner_{own} - Corner_{enemy} \linebreak + TrueLength_{own} \linebreak - FurthestField_{own} + FurthestField_{enemy} \linebreak + AccesibleSpace_{own} - AccesibleSpace_{enemy}$\\
		\pause
		Problem:\\
		$Bits \in \left[0,89\right]$\\
		$Corner \in \left[0,133\right)$\\
		$TrueLength \in \left[0,fieldlength * 2 \right)$\\
		$FurthestField \in \left[0,fieldlength * 2 \right)$\\
		$AccesibleSpace \in \left[0, fieldlength^2 \right)$\\
	}
	\only<4>{
		Lösung:\\
		$ (Bits_{own}) / Bits_{max}$ (=89)\\
		$ + (Corner_{own} - Corner_{enemy}) / Corner_{max}$ (=133)\\
		$ + (TrueLength_{own}) / (fieldlength * 2)$\\
		$ - (FurthestField_{own} + FurthestField_{enemy}) / (fieldlength * 2)$\\
		$ + (AccesibleSpace_{own} - AccesibleSpace_{enemy}) / (fieldlength^2)$
	}
\end{frame}
\begin{frame}
	$$ \frac{Bits_{own}}{Bits_{max}} * Bits_{weigth}$$\\
	$$ +\frac{Corner_{own} - Corner_{enemy}}{Corner_{max}}  * Corner_{weigth}$$\\
	$$ +\frac{TrueLength_{own}}{fieldlength * 2} * TrueLength_{weigth}$$\\
	$$ -\frac{FurthestField_{own} + FurthestField_{enemy}}{fieldlength * 2} * FurthestField_{weigth}$$\\
	$$ +\frac{AccesibleSpace_{own} - AccesibleSpace_{enemy}}{fieldlength^2} * AccesibleSpace_{weigth}$$
\end{frame}
\begin{frame}
	Aber wie findet man gute Gewichtungen?\\
	\begin{itemize}[<+(1)->]
		\item Alle gleich werten
		\item Gut nachdenken und mit Erfahrung
		\item Ausprobieren!
		\item Genetische Ai und viel Zeit
	\end{itemize}
\end{frame}

\subsection{...oder finden lassen}
\begin{frame}
	\note{Überleitung: ...und wo wir schon bei JS sind. JS is ja nicht dafür bekannt schnell zu sein.}
	\only<1>{Also starten und...}
	\only<2>{\includegraphics[width=1\linewidth]{media/wghnext.png}\\ \tiny Quelle: \url{http://cdn.windowsreport.com/wp-content/uploads/2016/08/aw-snap-google-chrome-windows-10.png}}
\end{frame}

\section{Optimierungen}
% \begin{itemize}[<+(1)->]
\subsection{Problem: MiniMax Tiefe?}
\begin{frame}
\note{Balance zwischen Rechenzeit und Stärke}
Welche Rekursionstiefe ist für den aktuelle Zustand die beste?\\
\pause
Mit fixer Tiefe:
\begin{itemize}
	\item Am Anfang sehr langsam
	\item Am Ende zu schwach
\end{itemize}
\pause
Array an fixer Tiefen:
\begin{itemize}
	\item Zu schwach in kritischen Situationen
	\item Wie vorher keine Kohärenz zum Verzweigungsfakor
\end{itemize}
\end{frame}
% hohe kosten für evaluation
\subsection{Lösung: Adaptiver MiniMax}
\begin{frame}
	Lösung: Adaptiver MiniMax\\
	\pause
	\centering
	\begin{tikzpicture}[sibling distance=5em,
	every node/.style = {shape=circle, draw, align=center }]]
	\node {}
	child { node { } }
	child { node { } }
	child { node { } }
	child { node { } };
	\end{tikzpicture}
\end{frame}
\begin{frame}[fragile]
	\begin{center}
		\begin{tikzpicture}[
		level 1/.style={sibling distance=5em},
		level 2/.style={sibling distance=1.2em},
		every node/.style = {shape=circle, draw, align=center }]]
		\node {}
		child { node { } child { node { } }child { node { } }child { node { } }child { node { } } }
		child { node { } child { node { } }child { node { } }child { node { } }child { node { } } }
		child { node { } child { node { } }child { node { } }child { node { } }child { node { } } }
		child { node { } child { node { } }child { node { } }child { node { } }child { node { } } };
		\end{tikzpicture}
	\end{center}
	\begin{flushleft}
		\pause
		Wiederhole solange $NodeCount_{curdepth} < AdaptiveMax$
		\pause
		\note{Aber warum current NodeCount}
		\\Das Verhältnis zwischen erstellen eines neuen GameStates und Evaluation davon ist etwa 1:10
	\end{flushleft}
\end{frame}
\begin{frame}

\end{frame}

% zacken diagramm
\subsection{Berechnungen wiederverwenden}
\begin{frame}
\end{frame}
\subsection{und notfalls löschen}
\begin{frame}
\end{frame}
\subsection{Multithreading (oder so)}
\begin{frame}
\end{frame}
\subsection{Inkrementelle GameStates}
\begin{frame}
\end{frame}

% + alpha/beta

\section{Mögliche Optimierungen (Ausblick)}
\subsection{Kleine Steine am Anfang ignorieren}
\begin{frame}
\end{frame}
\subsection{Dynamische Gewichtung}
\begin{frame}
\end{frame}
\subsection{Summen-Erreichbare Felder}
\begin{frame}
\end{frame}
% bla
% =======
\subsection{Anderer Algorithmus?}
\begin{frame}
\end{frame}
% Bild mit den zwei einsern
% Monte carlo ?
% sehr stark heuristik basierend
\subsection{Priorisieren von Gebieten in MiniMax}
\begin{frame}
\end{frame}
% Mehr Rechenleistung in interessante Gebiete stecken
\subsection{Neuschreiben, in einer besseren Sprache}
\begin{frame}
\end{frame}
% Performance, Memorymanagemenet, Threads

\end{document}